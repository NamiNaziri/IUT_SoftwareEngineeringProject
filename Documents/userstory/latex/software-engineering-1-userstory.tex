\documentclass[10pt,a4paper]{article}
\usepackage[margin=1in]{geometry}
\usepackage[utf8]{inputenc}
\usepackage[T1]{fontenc}
\usepackage{amsmath}
\usepackage{amsfonts}
\usepackage{amssymb}
\usepackage{graphicx}
\usepackage[hidelinks]{hyperref}
\usepackage[section]{placeins}
\usepackage{listings}
\usepackage{xcolor}
\usepackage{enumitem}
\usepackage{xepersian}

\settextfont{XB Niloofar}
\title{
بسم اللّه الرّحمن الرّحیم
~\vspace{0.5cm}
\begin{center}
	\includegraphics[width=0.2\linewidth]{assets/defaults/iut}
\end{center}
~\vspace{0.5cm}
دانشگاه صنعتی اصفهان
~\vspace{1cm}
\\
{
	\huge
\lr{user story}
 پروژه درس مهندسی نرم افزار ۱
\\~\\
\lr{GDM}
\\
\lr{(Game Development Manager)}
}
~\vspace{1cm}
}
\author{
محمد کسائی
,
آرین هادی
,
نامی نذیری
,
سید محمد غضنفری
}

\begin{document}
	\maketitle
	
\part{
دسترسی‌ها
}
\section*{
مدیریت افراد
}
\section*{
مدیریت تسک‌ها
}
\section*{
مدیریت‌ ددلاین‌ها
}
\section*{
مدیریت منابع
}

	\part{
\lr{Roles}	
}
\section*{
مدیر سیستم
}
تنظیمات سرور و تعریف اشتراک‌های مختلف ، دسترسی کامل به کاربران ثبت‌نام شده و امکان مسدود کردن کاربران را دارد.میزان منابع استفاده شده توسط هر پروژه در سرور‌ها را نیز مدیریت می‌کند.
\section*{
مدیر پروژه
}
تمام دسترسی‌ها نام‌برده شده را دارد.
\section*{
مدیر تیم‌
}
تمام دسترسی‌های نام برده شده محدود به تیم خودشان را دارند.
\section*{
	اعضای تیم
}
دسترسی‌های مدیریت تسک‌ها و ددلاین‌ها و منابع پروژه را برای شخص خودش مدیریت کند.
\section*{
کاربر عادی
}
در ابتدا به هیچ پروژه‌ای دسترسی ندارد ، می‌تواند اشتراک پروژه را خریداری کند.

\part{
	\lr{System Admin's User story}
}
\section*{
ورود 
}
مدیران سیستم (که از قبل هنگام برنامه نویسی به صورت hardcode اطلاعاتشان وارد سیستم شده است)  
میتوانند با اطلاعات کاربری خود وارد سیستم شوند و در صورت نیاز از سیستم بازیابی کلمه عبور (که توسط ارسال لینک به ایمیل انجام میشود) استفاده کنند.
\section*{
	پنل مدیریت
}
در اینجا مدیر سیستم میتواند دسترسی کامل به قسمت های مختلف سیستم داشته باشد و آن هارا مدیریت کند. که این قسمت ها شامل موارد زیر میباشند :
\begin{itemize}
	\item
	اضافه کردن شرکت هایی تحت عنوان "شرکت طرف قرارداد" ؛ که این شرکت ها میتوانند با عقد قرارداد های بلند مدتی با صاحبان و مدیران سیستم برای شرکتشان مزایا و قابلیت های ویژه ای را به صورت دائمی فراهم کنند . (این عمل میتواند به صورت اضافه کردن مدیران و کارمندان آن شرکت به سیستم و دادن دسترسی های به خصوص باشد)
	\item
	تعریف و مدیریت اشتراک های موجود در سیستم که میتواند شامل موارد زیر باشد:
	\begin{itemize}[label=$\star$]
		\item
		اضافه کردن اشتراک 
		\item
		حذف اشتراک های موجود
		\item
		مدیریت قیمت های اشتراک ها(مانند تعریف offer های زمانی یا کد تخفیف برای آن ها)
		
	\end{itemize}
	\item
	مدیریت کاربران سیستم (اضافه یا حذف کردن یا عوض کردن دسترسی های آن ها و ...)
	\item
	مدیریت پروژه های موجود در سیستم (مسدود کردن یا اضافه کردن و ...)
	\item
	قسمت آمار گیری (آمار کلی از کاربران ، شرکت ها و پروژه های موجود در سیستم و منابع استفاده شده توسط هر یک)
\end{itemize}

\part{
	\lr{Normal User's User story}
}
\section*{
	ثبت نام 
}
شخصی که برای اولین بار اقدام به استفاده از سیستم میکند با دادن اطلاعات زیر میتواند در سیستم ثبت نام کند :
\begin{itemize}
	\item 
	نام
	\item
	نام خانوادگی
	\item
	نام‌کاربری
	\item
	آدرس ایمیل
	\item
	رمزعبور و تکرار
\end{itemize}
پس از وارد کردن اطلاعات و صحت‌سنجی آن‌ها حساب کاربری ایجاد می‌شود برای فعال‌سازی ایمیل لازم است لینکی که به ایمیل ارسال شده است باز شود.

\section*{
	ورود
}
با وارد کردن نام‌کاربری و رمزعبور وارد حساب‌کاربری‌ خود می‌شوند. در صورتی که رمزعبور را فراموش کرده باشد باید نام‌کاربری را وارد کرده و فراموشی رمزعبور را انتخاب کند ایمیلی حاوی لینک بازیابی رمزعبور برای آدرس ایمیل ثبت‌شده در حساب‌کاربری متناظر با نام‌کاربری وارد شده ارسال می‌شود و کاربر می‌تواند رمز عبور خود را بازیابی کند

\section*{
	ویرایش اطلاعات کاربری
}
کاربر میتواند اطلاعات خود را تکمیل کند یا آن هارا تغییر دهد

\section*{
	خرید و مدیریت پروژه ها
}
کاربر میتواند با خرید اشتراک جدید یک پروژه جدید را برای خود آغاز کند (بدیهی است که پس از این اتفاق تبدیل به مدیر آن پروژه میشود) یا از بین پروژه های موجود انتخاب کند و وارد پنل آن ها شود(مدیر پروژه ، مدیر تیم یا عضو تیم) و پس از آن روند هر یک به صورت جداگانه در بخش های بعدی توضیح داده شده است


\part{
	\lr{Project Manager's User story}
}
در این بخش کاربری که نقش مدیر پروژه را دارد (که این اتفاق میتواند با خرید یک اشتراک توسط کاربر عادی و تبدیل شدن آن به مدیر پروژه رخ دهد یا این شخص از قبل طبق مواردی مانند "شرکت های طرف قرارداد "در سیستم تعریف شده باشد) مانند سایر کاربران وارد سیستم شده و پس از آن قسمت های زیر را در اختیار دارد:
\section*{
مدیریت اطلاعات کلی پروژه
}
در این قسمت مدیر پروژه میتواند اطلاعات کلی مربوط به پروژه را مشاهده و آن هارا ویرایش کند (مانند نام پروژه ، ارتقا و ویرایش پلن فعلی پروژه ، تعیین و تغییر پلتفرم های هدف پروژه و ... )

\section*{
	مدیریت تیم های موجود در پروژه
}
مدیر پروژه میتواند با کمک این قسمت تیم های جدیدی را تعریف کند یا تیم های موجود را مدیریت کند و افراد آن را کم یا زیاد کند یا نقش های مرتبط با آن هارا تغییر دهد
\section*{
	مدیریت تسک های موجود در پروژه
}
مدیر پروژه میتواند تسک های موجود هر تیم را مشاهده و آن هارا مدیریت کند (اضافه کم کردن یا تایین اولویت برای آن ها) یا طبق صلاحدید خود برای آن ها تسک های جدیدی را تعریف کند
\section*{
	مدیریت ددلاین های موجود در پروژه
}
در این قسمت میتوان ددلاین های موجود را مشاهده و آن هارا مدیریت کرد یا به آن ها ددلاین جدیدی را افزود
\section*{
مدیریت منابع موجود در پروژه
}
مدیر پروژه میتواند منابع حال حاضر پروژه را مشاهده و مدیریت کند
\section*{
مشاهده آمار و روند کلی مربوط پروژه
}
مشاهده روند کلی پروژه و روند پیشرفت تیم های مختلف (از نظر تسک های انجام شده و ددلاین های مورد انتظار)
\part{
	\lr{Team Manager's User story}
}
در این بخش کاربری که نقش مدیر تیم را دارد (که این اتفاق میتواند با محول شدن این نقش توسط مدیر پروژه به کاربر رخ دهد یا این شخص از قبل طبق مواردی مانند "شرکت های طرف قرارداد" در سیستم تعریف شده باشد) مانند سایر کاربران وارد سیستم شده و پس از آن قسمت های زیر را در اختیار دارد:
\section*{
	مدیریت افراد موجود در تیم
}
مدیر تیم میتواند اعضای تیم خود را مدیریت و به آن ها اضافه یا از آن ها فردی را کم کند
\section*{
	مدیریت تسک های موجود در تیم
}
مدیر تیم میتواند تسک های موجود  را مشاهده و آن هارا مدیریت کند (اضافه کم کردن یا تعیین اولویت برای آن ها) یا طبق صلاحدید خود تسک های جدیدی را تعریف کند
\section*{
	مدیریت ددلاین های موجود در تیم
}
در این قسمت میتوان ددلاین های موجود را مشاهده و آن هارا مدیریت کرد یا به آن ها ددلاین جدیدی را افزود
\section*{
	مدیریت منابع در دسترس تیم
}
مدیر تیم میتواند منابع در دسترس تیم را مدیریت کند و موارد نهایی ساخته شده توسط تیم خود را در منبع اصلی پروژه قرار دهد 
\section*{
مشاهده آمار و روند کلی مربوط تیم 
}
مشاهده روند کلی پیشرفت تیم شامل تسک های انجام شده یا ددلاین های موجود یا عملکرد افراد تیم
\part{
	\lr{Team Member's User story}
}
در این بخش کاربری که عضو تیم است (که این اتفاق میتواند با محول شدن این نقش توسط مدیر پروژه یا مدیر یک تیم به کاربر رخ دهد یا این شخص از قبل طبق مواردی مانند "شرکت های طرف قرارداد" در سیستم تعریف شده باشد) مانند سایر کاربران وارد سیستم شده و پس از آن قسمت های زیر را در اختیار دارد:
\section*{
	مشاهده تسک های محول شده
}
در این قسمت میتوان تسک های محول شده و اولویت آن ها را مشاهده کرد و نسبت به انجام آن ها اقدام کند یا برای خود تسک های جدیدی را تعریف کند
\section*{
	مشاهده ددلاین ها
}
در این قسمت میتوان ددلاین های مربوط به خود را مشاهده کند یا برای خود ددلاین جدیدی را تعریف کند
\section*{
	کار با منابع در دسترس تیم
}
هر شخص میتواند از منابع در دسترس تیم استفاده کند

\end{document}

