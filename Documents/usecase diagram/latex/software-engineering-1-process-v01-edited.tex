\documentclass[10pt,a4paper]{article}
\usepackage[margin=1in]{geometry}
\usepackage[utf8]{inputenc}
\usepackage[T1]{fontenc}
\usepackage{amsmath}
\usepackage{amsfonts}
\usepackage{amssymb}
\usepackage{graphicx}
\usepackage{hyperref}
\usepackage[section]{placeins}
\usepackage{listings}
\usepackage{xcolor}
\usepackage{xepersian}

\settextfont{XB Niloofar}

\title{
	بسم اللّه الرّحمن الرّحیم
	~\vspace{0.5cm}
	\begin{center}
		\includegraphics[width=0.2\linewidth]{assets/defaults/iut}
	\end{center}
	~\vspace{0.5cm}
	دانشگاه صنعتی اصفهان
	~\vspace{1cm}
	\\
	{
		\huge
		فرایند پروژه درس مهندسی نرم افزار ۱
		\\~\\
		\lr{GDM}
		\\
		\lr{(Game Development Manager)}
	}
	~\vspace{1cm}
}

\author{
محمد کسائی
,
آرین هادی
,
نامی نذیری
,
سید محمد غضنفری
}
\begin{document}
	\maketitle
	\newpage
	در شرح فرایندها در صورتی که اجرا کننده‌ی مرحله ذکر نشود همان شخصی که وارد سامانه شده است مدنظر است.
	\section{
مدیریت پروژه
}
\subsection{
ایجاد پروژه
}
\begin{enumerate}
	\item 
	کاربر عادی وارد سیستم می‌شود.
	\item
	نام پروژه را انتخاب می‌کند.
	\item
	نوع پروژه و اشتراک مناسب را انتخاب می‌کند.
	\item
	پرداخت لازم برای اشتراک انتخاب شده انجام می‌شود.(در صورتی که فرد قبلا توسط مدیر سیستم تعریف شده باشد نیازی به پرداخت نیست.)
	\item
	از این پس این کاربر به عنوان مدیر این پروژه شناخته می‌شود.
\end{enumerate}

\subsection{
	ویرایش پروژه
}

\begin{enumerate}
	\item 
	مدیر پروژه وارد می‌شود.
	\item
	گزینه‌ی ویرایش پروژه‌ را انتخاب می‌کند.
	\item
	 مشخصات پروژه نمایش داده می‌شود.
	 \begin{itemize}
	 	\item 
	 	نام پروژه
	 	\item
	 	توضیحات پروژه
	 	\item
	 	ددلاین کلی پروژه ( این ددلاین مستقل از ددلاین‌های تسک‌های تعریف شده است.)
	 \end{itemize}
 \item
 ویرایش‌های لازم را انجام می‌دهد.
 \item
 با تایید او ویرایش‌ها در سیستم ثبت می‌شود.
 \item
 در صورت نیاز به هر یک از افراد دخیل در پروژه پیامی مبنی بر تغییر قسمتی از مشخصات پروژه ارسال می‌شود.
\end{enumerate}

\subsection{
بررسی وضعیت پیشرفت پروژه
}


\begin{enumerate}
	\item 
	مدیر پروژه وارد می‌شود.
	\item
	گزینه‌ی بررسی پروژه را انتخاب می‌کند.
	\item
	می‌تواند نمودار‌های مختلفی از روند پیشرفت پروژه را مشاهده کند.
	\item
	می‌تواند از نمودار‌ها خروجی بگیرد.
	\item
	نمودارهای زیر را می‌بیند.
		 \begin{itemize}
		\item 
		تسک‌ها با توجه به وضعیت آن‌ها و تیم‌های وابسته
		\item
		وضعیت کلی ددلاین‌ها و تعداد ددلاین‌های گذشته از موعد
	\end{itemize}
	
\end{enumerate}


\section{
مدیریت تیم
}

\subsection{
ایجاد تیم جدید
}
\begin{enumerate}
	\item 
	مدیر پروژه وارد می‌شود.
	\item
	گزینه‌ی افزودن تیم جدید را انتخاب می‌کند.
	\item
	نام تیم را مشخص می‌کند.
	\item
	مدیر تیم را مشخص می‌کند.
	\item
	در صورت نیاز افرادی را هم به پروژه اضافه کند.
	\item
	به تمام افراد اضافه شده به تیم پیامی ارسال می‌شود.
\end{enumerate}

\subsection{
	ویرایش تیم (مدیر پروژه)
}

\begin{enumerate}
	\item 
	مدیر پروژه وارد می‌شود.
	\item
	یک تیم را انتخاب میکند.
	\item
	گزینه‌ی ویرایش تیم را انتخاب می‌کند.
	\item
	مشخصات کلی تیم را مشاهده می‌کند.
	\begin{itemize}
		\item 
		عنوان تیم 
		\item
		مدیر تیم
		\item
		توضیحات و شرح وظایف تیم
		\item
		افراد تیم 
	\end{itemize}
	\item
	هر یک از ويژگی‌ها را به دلخواه ویرایش می‌کند.
	\item
	با تایید او تغییرات در سیستم ثبت می‌شود.
	\item
	در صورت نیاز به کاربران عضو تیم اطلاع داده شود.
	
\end{enumerate}


\subsection{
	ویرایش تیم (مدیر تیم)
}

\begin{enumerate}
	\item 
	مدیر تیم وارد می‌شود.
	\item
	گزینه‌ی ویرایش تیم را انتخاب می‌کند.
	\item
	مشخصات کلی تیم را مشاهده می‌کند.
	\begin{itemize}
		\item 
		عنوان تیم 
		\item
		مدیر تیم
		\item
		توضیحات و شرح وظایف تیم
		\item
		افراد تیم 
	\end{itemize}
	\item
	هر یک از ويژگی‌ها را به دلخواه ویرایش می‌کند.
	\item
	با تایید او تغییرات در سیستم ثبت می‌شود.
	\item
	در صورت نیاز به کاربران عضو تیم اطلاع داده شود.
	
\end{enumerate}

\subsection{
حذف تیم
}

\begin{enumerate}
	\item 
	مدیر پروژه وارد می‌شود.
	\item
	گزینه‌ی حذف تیم را انتخاب می‌کند.
	\item
	با تایید او حذف انجام می‌شود.
	\item
	به افرادی که در این تیم مشارکت داشته‌اند پیامی ارسال می‌شود.
\end{enumerate}


	\section{
	مدیریت تسک
}
\subsection{
افزودن تسک (مدیر پروژه)
}
\begin{enumerate}
	\item 
	 مدیر پروژه وارد می‌شود.
	 \item
	 یک تسک جدید ایجاد می‌کند.
	 \item
	 عنوان تسک جدید را انتخاب می‌کند
	 \item
	 گروه‌های دخیل در این تسک را انتخاب می‌کند.
	 \item
	 ممکن است ددلاین تسک را مشخص کند.
	 \item
	 با تایید مدیر پروژه تسک ایجاد می‌شود.
	 \item
	 به تمام افراد مرتبط با تسک ایجاد شده پیامی ارسال می‌شود.
\end{enumerate}

\subsection{
	ویرایش تسک (مدیر پروژه)
}

\begin{enumerate}
	\item 
	مدیر پروژه وارد می‌شود.
	\item
	یک تسک موجود را انتخاب می‌کند.
	\item
	می‌تواند گروه‌های مرتبط را تغییر دهد.
	\item
	ممکن است ددلاین تسک را تغییر بدهد.
	\item
	با تایید مدیر پروژه تسک ویرایش می‌شود.
	\item
	در صورت نیاز به افراد دخیل پیامی ارسال می‌شود.
\end{enumerate}



\subsection{
	افزودن تسک (مدیر تیم)
}
\begin{enumerate}
	\item 
	مدیر تیم وارد می‌شود.
	\item
	یک تسک جدید در محدوده‌ی تیم خودش ایجاد می‌کند.
	\item
	عنوان تسک جدید را انتخاب می‌کند
	\item
	می‌تواند این تسک را به افرادی از تیم خودش اختصاص دهد.
	\item
	ددلاین تسک را مشخص می‌کند.
	\item
	با تایید او تسک ایجاد می‌شود.
	\item
	به تمام افراد مرتبط با تسک ایجاد شده پیامی ارسال می‌شود.
\end{enumerate}
\subsection{
	کامل کردن تسک (مدیر تیم)
}

\begin{enumerate}
	\item 
	مدیر تیم وارد می‌شود.
	\item
	یک تسک موجود در تیم خودش را انتخاب می‌کند.
	\item
	در صورت نیاز توضیح اضافه‌ای می‌نویسد.
	\item
	مدیر پروژه این تسک را بررسی می‌کند.
	\item
	در صورتی که تسک از نظر مدیر پروژه صحیح باشد تکمیل شده اعلام می‌شود و به افراد دخیل در تسک پیامی ارسال می‌شود.
	\item
	در صورتی که کامل شدن توسط مدیر تایید نشد پیامی به افراد دخیل در تسک ارسال می‌شود و هم‌چنان باز می‌ماند.
	
\end{enumerate}


\subsection{
	ویرایش تسک (مدیر تیم)
}

\begin{enumerate}
	\item 
	مدیر تیم وارد می‌شود.
	\item
	یک تسک موجود در تیم خودش را انتخاب می‌کند.
	\item
	می‌تواند افراد مرتبط با این تسک را تغییر دهد.
	\item
	ممکن است ددلاین تسک را تغییر بدهد.
	\item
	با تایید او تسک ویرایش می‌شود.
	\item
	در صورت نیاز به افراد دخیل پیامی ارسال می‌شود.
\end{enumerate}

\subsection{
افزودن تسک (اعضای تیم)
}
\begin{enumerate}
	\item
	هر فردی در نقش عضو تیم وارد می‌شود.
	\item
	می‌تواند براساس تسک‌های تعریف شده توسط مدیر‌های رده‌بالاتر تسک اختصاصی خودش را این تعیین کند.
	\item
	این تسک فقط و فقط به خود او اختصاص می‌یابد.
	\item
	عنوان تسک را مشخص می‌کند.
	\item
	زمان مشخصی برای انجام آن تعیین می‌کند.
	\item
	اضافه شدن این تسک به مدیر تیم اطلاع‌رسانی می‌شود.
\end{enumerate}

\subsection{
	کامل کردن تسک (اعضای تیم)
}
\begin{enumerate}
	\item
	هر فردی در نقش عضو تیم وارد می‌شود.
	\item
	می‌تواند تسک هایی را که خودش تعریف کرده است انتخاب کند.
	\item
	در صورت نیاز توضیح اضافه‌ای بنویسد.
	\item
	پیامی به مدیر تیم ارسال می‌شود.
	\item
	مدیر تیم باید نحوه‌ی انجام را بررسی کند. با تشخیص او ممکن است تسک به درستی کامل شده باشد که در این‌صورت پیامی به عضو تیم ارسال می‌شود و تسک به عنوان کامل شده در سیستم ثبت می‌شود.
	\item
	در صورتی که به نظر مدیر تیم درست نباشد پیامی به عضو تیم ارسال می‌شود و آن تسک باز خواهد ماند.
	
\end{enumerate}

\section{
	مدیریت ددلاین‌ها
}
این بخش مربوط به ددلاین‌های کلی و شامل تعدادی تسک مختلف است و ددلاین‌های ساده‌ی مربوط به یک تسک در بخش‌های قبلی ذکر شده‌اند.
\subsection{
افزودن ددلاین 
}
\begin{enumerate}
	\item 
	مدیر پروژه یا مدیر تیم وارد می‌شود.
	\item
	عنوان مناسبی برای ددلاین انتخاب می‌شود.
	\item
	مجموعه‌ی تسک‌های مرتبط با هم را انتخاب می‌کند.
	\item
	پیامی برای تمام افراد دخیل در تسک‌های انتخاب شده ارسال می‌شود.
\end{enumerate}
\section{
	مدیریت کاربران
}

\subsection{
	ورود کاربر
}

\begin{enumerate}
	\item 
	نام کاربری و رمز عبور خود را وارد می‌کند.
	\item
	به ایمیل ثبت شده کد یک‌بار مصرف ارسال می‌شود.
	\item
	کاربر باید کد را وارد کند.
	\item
	پس از بررسی صحت کد فرایند ورود به سیستم کامل می‌شود.
	
\end{enumerate}

\end{document}

