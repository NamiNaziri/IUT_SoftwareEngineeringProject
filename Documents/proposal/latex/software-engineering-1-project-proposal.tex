\documentclass[10pt,a4paper]{article}
\usepackage[margin=1in]{geometry}
\usepackage[utf8]{inputenc}
\usepackage[T1]{fontenc}
\usepackage{amsmath}
\usepackage{amsfonts}
\usepackage{amssymb}
\usepackage{graphicx}
\usepackage[hidelinks]{hyperref}
\usepackage[section]{placeins}
\usepackage{listings}
\usepackage{xcolor}
\usepackage{xepersian}

\settextfont{XB Niloofar}
\title{
بسم اللّه الرّحمن الرّحیم
~\vspace{0.5cm}
\begin{center}
	\includegraphics[width=0.2\linewidth]{assets/defaults/iut}
\end{center}
~\vspace{0.5cm}
دانشگاه صنعتی اصفهان
~\vspace{1cm}
\\
{
	\huge
پروپوزال پروژه درس مهندسی نرم افزار ۱
\\~\\
\lr{GDM}
\\
\lr{(Game Development Manager)}
}
~\vspace{1cm}
}
\author{
محمد کسائی
,
آرین هادی
,
نامی نذیری
,
سید محمد غضنفری
}
\begin{document}
	\maketitle
	\newpage
	\tableofcontents
	\newpage
	
	
	\part{
	برتری پروژه از جنبه‌های مختلف
}
\section{
	بررسی موضوع
}
\subsection{
توضیح کلی
}
طراحی و پیاده‌سازی سیستم مدیریت پروژه های بازیسازی برای تیم های کوچک تا بزرگ به نحوی که بازیسازان مختلف در تیم هایی با ابعاد مختلف بتوانند صفر تا صد پروژه ی خود را با این سیستم پیش ببرند و قادر باشند تا برای قسمت های مختلف مورد نیاز در یک پروژه‌ی بازیسازی با هم دیگر همکاری کنند و پیشرفت این قسمت ها را دنبال کنند یا منابع مختلف را با یک دیگر به اشتراک بگذارند.

\subsection{
	ضرورت انجام پروژه
}
پروژه های بازیسازی به طور معمول پروژه هایی بزرگ و دارای قسمت های مختلف و متنوع میباشند که هر قسمت ویژگی های خاص خود و افراد متخصص خود را دارا میباشد. برای ساخته شدن محصول نهایی نیاز است که این قسمت ها در هماهنگی کامل با یکدیگر کار کنند و از نتایج کار یکدیگر برای محصول نهایی استفاده کنند به همین علت نیاز به یک پلتفرم برای هماهنگ سازی این قسمت ها و اشتراک نتایج و منابع بین آن ها در این گونه پروژه ها غیر قابل انکار است.

\subsection{
	اهداف پروژه 
}
کمک به پروژه های بازیسازی در سطوح مختلف برای پیشبرد پروژه های خود با مشخص کردن قسمت های مختلف و به اشتراک گذاری نتایج و منابع بین آن ها برای ساده سازی و سرعت بخشیدن و نظم بخشیدن به روند ساخت بازی.

\subsection{
	خدمات قابل ارائه
}
\begin{itemize}
	\item
	تعیین و بوجود آوردن قسمت های مختلف دخیل در یک پروژه بازیسازی بنا به نوع پروژه (کوچک و مستقل یا بزرگ و جامع)
	\item
	مشخص کردن ددلاین برای قسمت های محتلف
	\item
	تعریف افراد هر قسمت و دادن دسترسی و محول سازی وظایف به هر یک
	\item
	به اشتراک گذاری منابع مورد نیاز برای قسمت های مختلف (\lr{texture} ها یا \lr{art} های هنری یا \lr{track} های موسیقی و ...)
\end{itemize}

\subsection{
	استفاده کنندگان اصلی نرم‌افزار
}
این نرم افزار برای تمامی پروژه های بازیسازی اعم از پروژه های یک نفره و پروژه های مستقل چند نفره تا پروژه های بزرگ و جامع که دارای استدیو ها و شعبات مختلف هستند ، میتواند کاربرد داشته باشد.

\section{
سیستم‌های مشابه و نحوه‌ی سرویس‌دهی
}
\begin{itemize}
	\item 
	\lr{: HelixCore + Hansoft (Perforce)}
	\\
	این پکیج پیشنهادی شرکت 
	\lr{Perforce}
	برای پروژه های بازیسازی میباشد که شامل دو نرم افزار یکی برای ورژن کنترل و دیگری برای مدیریت پروژه است.
	\item
	\lr{: codecks.io}
	\\
	این سایت خود بر مبنای بازی های کارتی طراحی شده که در آن وظایف و کار ها در قالب دست های مختلف و دسته به اعضا محول میشود.
	\item
	\lr{: hacknplan.com}
	\\
این سایت قسمت های مختلف تیم را مدیریت میکند و برای آنها تایم لاین زمانی مشخص درست میکند.
\end{itemize}

\section{
نحوه‌ی انجام پروژه
}
تقسیم بندی پروژه به قسمت های کوچکتر و الویت بندی آن ها و پیاده سازی هر قسمت، که این قسمت ها میتواند به شرح زیر باشد : 
\begin{itemize}
	\item
	مدیریت کاربران (اعم از تیم ها و مدیران)
	\item
	تعریف کردن انواع حساب های کاربری برای تیم های مختلف(مخصوص به بازیساز های تک نفره،مستقل یا بزرگ)
	\item
تعریف نقش های مختلف در هر پروژه
\item
	پیاده سازی قسمت وظایف موجود در هر پروژه 
	\item
	پیاده سازی قسمت ددلاین ها
	\item
	پیاده سازی قسمت ورژن کنترل و یا متصل کردن به سیستم های معروف مانند گیت هاب
	\item
	پیاده سازی قسمت ارتباط افراد دخیل در پروژه (مانند قسمت چت اعضا)
	\item
	پیاده سازی قسمت اشتراک منابع در پروژه
\end{itemize}

\section{
توجیه اقتصادی
}
باید در نظر داشت که صنعت بازیسازی یکی از ۳ صنعت پردرآمد حال حاضر دنیا میباشد و به صورت روزافزون رشد میکند و هر روز تیم های کوچک و بزرگ بسیاری از سرتاسر دنیا به این صنعت افزوده میشوند به همین علت نیاز به یک برنامه‌ی مدیریت پروژه های بازیسازی برای این تیم ها یک نیاز ضروری میباشد که آن هارا مجبور به پرداخت هزینه و استفاده از این ابزار میکند و چون این صنعت یک صنعت پردرآمد است و نظم و سرعت برای ساخت بازی ها بسیار حیاتی است اکثر تیم ها حاضر به پرداخت هزینه برای نرم افزارهایی با قیمت های مناسب و مزایای رقابتی برای پیشبر ساخت بازی های خود خواهند بود.\\
ما در نظر داریم این برنامه را به صورت اشتراک های مختلف (از نظر امکانات و زمان) ارائه کنیم تا تیم های بازیسازی با خرید این اشتراک ها از نرم افزار ما استفاده کنند و برنامه به سوددهی برسد .

\section{
تاثیر در آینده‌ی کاری
}
ساخت این پروژه میتواند به تیم ما کمک کند تا با صنعت بازیسازی و نحوه‌ی ساخت بازی ها آشنا شوند و از آنجایی که در ساخت بازی ها اغلب افراد با تخصص های مرتبط با رشته‌ی کامپیوتر درگیر هستند میتواند به تیم ما یک دید بهتر نسبت به این 
صنعت و نحوه‌ي ورود به آن بدهد.
\\
همچنین پیاده سازی این پروژه به مهارت های برنامه نویسی و طراحی نرم افزار های چندقسمتی و بزرگ اعضای تیم می افزاید.

\section{
بررسی تبادل دانش دوطرفه بین تیم و مشتری
}
این پروژه برای مشتری به خصوصی  ساخته نمیشود بلکه با توجه به نیاز های موجود در صنعت بازیسازی ساخته خواهد شد و پس از آن به صورت فروش اشتراک ، مشتری های خود را پیدا میکند که با زیاد شدن تعداد آن ها میتوان از ایده ها و نظرات آن ها بهره مند شد و برنامه را گسترش داد.

\section{
قابلیت‌های تیم 
}
تیم ما دارای افراد با سابقه در بخش های مورد نیاز این پروژه مانند آشنایی با صنعت بازیسازی و نحوه‌ی ساخت بازی ها و همچنین دارای قابلیت ها و تجربه های طراحی نرم افزار به خصوص طراحی UI و طراحی وبسایت میباشد . علاوه بر این موارد، به علت آشنایی افراد تیم با زبان های برنامه نویسی مختلف باعث میشود در پیاده سازی قسمت های مختلف پروژه ابزار های متنوع و قدرتمند بسیاری را دارا باشیم.
	\newpage
\part{
میزان تاثیر پروژه
}
\section{
قابلیت انتشار پروژه به صورت مقاله
}
از آنجایی که این نرم افزار بیشتر به صورت کاربردی و عملی میباشد دارای قسمت های علمی قابل ارائه در مقاله نمیباشد.
\section{
روش ارائه‌ی نتایج کار به گروه‌های هدف
}
این پروژه به صورت سایت پیاده خواهد شد و این سایت را میتوان در جوامع مختلف بازیسازی که مخاطب بالا دارند(مانند نمایشگاه های سالانه معروف مانند
\lr{GDC}
) تبلیغ نمود و آن ها را به خرید اشتراک و استفاده از این سیستم تشویق کرد.
	\newpage
\part{
پیاده‌سازی
}

\section{
برنامه‌ی احتمالی برای پیاده‌سازی
}
پس از مشخص شدن کامل قسمت های مختلف پروژه باید قسمت های مختلف را موازی پیش برد و آماده کرد.
\lr{(parallel~development)}
از آنجایی که این پروژه در قالب سایت و سرویس های ابری پیاده سازی میشود میتوان قسمت پیاده سازی آن را به دو قسمت اصلی 
\lr{frontend}
 و
 \lr{backend}
 تقسیم نمود و اعضای تیم میبایست بنا به توانایی هایشان به صورت موازی بر روی هر یک از این قسمت ها مشغول به کار شوند و یک نفر هم باید پیشرفت کلی هر دوقسمت و هماهنگ سازی بین این دو قسمت را مدیریت کند تا در انتها با ادغام نتایج این دو گروه پروژه به اتمام برسد.\\
  لازم به ذکر است که این پروژه خاصیت فاز بندی هم میتواند داشته باشد چرا که میتوان قسمت های اصلی تر و مورد نیاز تر این سیستم را ابتدا پیاده کرد و پس از انتشار و گرفتن بازخورد و رفع مشکلات احتمالی قسمت های جزیی تر را به سیستم در نسخه های آینده اضافه نمود.\\
 به صورت خلاصه میتوان سه قسمت نام برده شده در پیاده سازی را این گونه شرح داد:
\begin{itemize}
	\item 
	{
\large	
\lr{: frontend}
}
باید ابتدا رابطه کاربری و اجزای موجود در صفحه را طراحی کرد و پس از آن میتوان قسمت frontend را به صورت مستقل و کامل پیاده سازی کرد.
	\item 
{
	\large	
	\lr{: backend}
}
این قسمت وظایف سنگین تری دارد و در بعضی از بخش های پیاده سازی این قسمت نتایج بخش های دیگر مورد نیاز است یا به یکدگیر مربوط است.به طور کلی روند پیاده سازی آن میتواند به صورت پیاده سازی API های زیر باشد :
\begin{enumerate}
	\item
	ابتدا باید قسمت مدیریت کاربران سایت طراحی شود (کاربران عادی و مدیران و امکانات پنل های آنها)
	\item
	طراحی پنل های مورد نیاز برای انواع کاربران و ویژگی ها و امکانات آن ها
	\item
	تعریف کردن نقش های مختلف در کاربران نوع جمعی
	\item 
	تعریف کردن و ساخت قسمت وظایف مورد نیاز در پروژه های بازیسازی مختلف
	\item
	ساخت و تعریف قسمت ددلاین ها
\end{enumerate}
	موارد قبلی زیربنای اصلی (فاز اصلی و اول) برنامه ی ما میباشند و پیاده سازی همگی آنها با ترتیب گفته شده برای نتیجه‌ی اصلی و نهایی مورد نیاز هستند ولی قسمت های ورژن کنترل ، قسمت ارتباطات و چت و قسمت اشتراک منابع را میتوان پس پایان قسمت قبلی و ادغام با UI آن اضافه نمود.(فاز های جانبی بعدی)
	


	\item 
{
	\large	
	\lr{: integration and testing}
}
در این قسمت UI پیاده سازی شده با API های توضیح داده شده در انتهای پیاده سازی هر فاز به یکدیگر متصل میشوند (توسط کل اعضای تیم) سپس عملکرد کلی سایت مورد آزمایش قرار میگیرد. ‌
\end{itemize}
\section{
ریسک‌های احتمالی و برطرف نمودن خطرات
}	

\begin{enumerate}
	\item 
	{
\large
عدم استقبال	

}
	از آنجایی که این پروژه به صورت سفارشی ساخته نمیشود و مبنای آن فروش اشتراک پس از ساخته شدن با هزینه های شخصی تیم است در صورت عدم استقبال تیم را متوجه ضرر های مادی و معنوی میکند. \\
	\textbf{راه برطرف سازی : }
	میتوان پیش از انجام پروژه با افراد مختلف مربوط با این صنعت صبحت نمود و میزان علاقه و استقبال آن هارا متوجه شد یا راه های بهتر پیاده کردن پروژه برای استقبال بیشتر را پیدا کرد.
	\item
	{
\large
مشکلات ارزی	

}
در صورتی که این پروژه در یک تیم ایرانی و با محدودیت های ایرانی بخواهد پیاده سازی شود به طبع فروش آن به تیم های خارحی و دریافت هزینه های مربوطه از آن ها یکی از ریسک های پیاده سازی پروژه خواهد بود.
\\
\textbf{راه برطرف سازی :} 
داشتن حساب و رابط های خارجی برای انتقال ارز به داخل کشور یا استفاده از ارزهای دیجیتال. 
\item
{
\large
مشکلات فنی

}
از آنجایی که این پروژه در قالب سایت آماده میشود باید مشکلات مربوط به این نوع پیاده سازی را در نظر گرفت مانند مشکلات فنی که ممکن است برای سرور ها پیش بیاید (سربار ترافیکی روی سرور ها و کمبود منابع برای مدیریت ترافیک و ...) یا مشکلات مربوط به امنیت برای حفظ منابع و اسرار تیم های استفاده کننده از سایت.\\
\textbf{راه برطرف سازی :}
دقت در انتخاب منابع و مدیریت درست آنها و توجه به مسائل امنیتی باید برای رفع این گونه مشکلات در نظر گرفته شود.
\end{enumerate}
\end{document}

